\documentclass[runningheads,a4paper]{llncs}
%\documentclass[runningheads,a4paper]{article}
\thispagestyle{empty}


%\usepackage[T1]{fontenc}
%\usepackage{calligra}
%\usepackage{verbatim}
%\usepackage{longtable}
%\usepackage{stmaryrd}
%http://www.phil.cam.ac.uk/teaching_staff/Smith/logicmatters/l4llogiciansnew.html
%http://www.phil.cam.ac.uk/teaching_staff/Smith/LaTeX/guides/BussGuide2.pdf
%\usepackage{bussproofs}
%ftp://ftp.ams.org/pub/tex/doc/amsmath/short-math-guide.pdf
\usepackage{amssymb}
\usepackage{amsmath}

\usepackage{graphicx}
%\usepackage{pstricks}
\usepackage{ifthen}

%llncs
\providecommand{\institute}[1]{\gdef\@institute{#1}}

%% atomic strings
\def\id#1{\text{\em #1\/}}
\def\relid#1{\ifthenelse{\equal{#1}{r}
                      \or\equal{#1}{s}
                      \or\equal{#1}{t}
                      \or\equal{#1}{R}
                      \or\equal{#1}{S}
                      \or\equal{#1}{T}}{#1}
                                       {\id{#1}}}
\def\cpt#1{\ifthenelse{\equal{#1}{A}
                    \or\equal{#1}{B}
                    \or\equal{#1}{C}}{#1}
                                     {\text{\tt #1\/}}}
\def\atom#1{\ifthenelse{\equal{#1}{a}
                     \or\equal{#1}{b}
                     \or\equal{#1}{c}}{#1}
                                      {\text{\tt '#1'\/}}}
\newcommand{\begrip}[1]{\emph{#1}}
%% common symbols
\newcommand{\la}{\langle}
\newcommand{\ra}{\rangle}
\newcommand{\powerset}[1]{\mathcal P(#1)}
\newcommand{\subs}{\subseteq}
\newcommand{\flip}[1]{{#1}^\smallsmile}
\newcommand{\cmpl}[1]{\overline{#1}}
\newcommand{\compose}{;}
\newcommand{\union}{\cup}
\newcommand{\intersect}{\cap}
\newcommand{\iden}{\mathbb{I}}
\newcommand{\full}{\mathbb{V}}
%\newcommand{\relAdd}{\dagger}
%\newcommand{\kleeneplus}[1]{{#1}^+}
%\newcommand{\kleenestar}[1]{{#1}^*}
%\newcommand{\rewriteto}{\rightarrow}
%\newcommand{\calc}{\implies}
%\newcommand{\anything}{\top}
%\newcommand{\nothing}{\bot}
%\newcommand{\fun}{\rightarrow}
%% Ampersand expressions
\newcommand{\type}[2]{\cpt{#1}\sim\cpt{#2}}
\newcommand{\htype}[1]{\cpt{#1}}
\newcommand{\fulltype}[2]{\cpt{#1}\times\cpt{#2}}
\newcommand{\expr}[3]{{#1}_{\type{#2}{#3}}}
\newcommand{\hexpr}[2]{{#1}_{\ifthenelse{\equal{#2}{X}\or\equal{#2}{Y}}{#2}{\htype{#2}}}}
\newcommand{\fullexpr}[3]{{#1}_{\fulltype{#2}{#3}}}
\newcommand{\declare}[3]{\relid{#1}:\type{#2}{#3}}
\newcommand{\applyRel}[3]{\atom{#2}\ \relid{#1}\ \atom{#3}}
\newcommand{\pair}[2]{\la\atom{#1}, \atom{#2}\ra}
%% Ampersand symbols
\newcommand{\atoms}{\mathbb{U}}
\newcommand{\decls}{\mathbb{D}}
\newcommand{\rels}{\mathbb{R}}
\newcommand{\cpts}{\mathbb{C}}
\newcommand{\context}{\mathfrak{C}}
\newcommand{\rul}{\id{RUL}}
\newcommand{\rel}{\id{REL}}
\newcommand{\pop}{\id{POP}}
\newcommand{\siden}[1]{\hexpr{I}{#1}}
\newcommand{\sfull}[2]{\fullexpr{V}{#1}{#2}}
\newcommand{\isa}{\epsilon}
\newcommand{\sflip}[1]{{#1}^\sqcup}
\newcommand{\scmpl}{\lnot}
\newcommand{\scompose}{\comp}
%copied from z-eves.sty
\def\comp{\mathrel{\raise 0.66ex\hbox{\oalign{\hfil%
	  $\scriptscriptstyle\mathsf{o}$\hfil%
	  \cr\hfil$\scriptscriptstyle\mathsf{9}$\hfil}}}}
\newcommand{\sunion}{\sqcup}
\newcommand{\sintersect}{\sqcap}
\newcommand{\simplic}{\Rightarrow}
\newcommand{\sequiv}{\equiv}
%% Ampersand functions
\newcommand{\tf}[1]{\mathfrak{T}(#1)}
\newcommand{\ptf}[1]{\mathfrak{T}'(#1)}
\newcommand{\ti}[1]{\mathfrak{I}(#1)}
%% feedback message
\newcommand{\msg}[1]{\begin{center}\emph{#1}\end{center}}

\usepackage{url}

\hyphenation{add-ing}

\newcommand{\keywords}[1]{\par\addvspace\baselineskip\noindent\keywordname\enspace\ignorespaces#1}

\begin{document}
\section*{Demonstration of information system prototype derived from relational specification}
\subsubsection{Navigate the conceptual model}
See the conceptual diagrams, figures 3.1, 3.2 and 4.2, in the functional specification of the context \id{VIRO}.
Concepts are vertices; relations are edges; relationships are not represented.

\subsubsection{Edit relationships}
\begin{eqnarray}
 &\declare{caseFile}{Document}{LegalCase} \\ 
 &\applyRel{caseFile}{doc987384}{199902238} \\
 &\applyRel{caseFile}{doc763820}{AWB 07/2481 WRO} \\
 &\applyRel{caseFile}{letter 2009/87743}{SBR 02/74331} \\
 &\applyRel{caseFile}{letter 2009/87743a}{SBR 02/74331} \\
 &\applyRel{caseFile}{schedule 2009/87743.1}{SBR 02/74331} 
\end{eqnarray}

$\applyRel{caseFile}{doc763820}{AWB 07/2481 WRO}$ has the purpose to mean: \emph{doc763820 is a document in the case file of case awb 07/2481 wro}.

\subsubsection{Maintain invariant rules}
The next law text is used to demonstrate the prevention of a violation of this juristic rule.

\begin{quote}
\emph{An appeal lodged against a decision of an administrative authority of a province or municipality, or a water management board, or a region as referred to in article 21 of the 1993 Police Act, or of a joint body or public body established under the Joint Arrangements Act, falls within the jurisdiction of the district court within whose district the administrative authority has its seat. (art. 8:7 par.1 Awb.)}
\end{quote}

Invariant rules are relations declared as rules.
Properties of relations, e.g. univalence, are invariant rules which have a shorthand syntax.

\begin{eqnarray}
	&\declare{appeal}{LegalCase}{LegalCase}\id{(symm,antisym)} \\
	&\declare{defendant}{Party}{LegalCase} \\
	&\declare{adminAuthAwb87}{Party}{Party}\id{(symm,antisym)} \\
	&\declare{domicile}{Party}{City}\id{(univ)} \\
	&\declare{jurisdiction}{City}{Court}\id{(univ,total)} \\
	&\declare{broughtBefore}{LegalCase}{Court} \\
	&\cpt{RULE}\ \relid{appeal}\scompose\sflip{\relid{defendant}}\scompose\relid{adminAuthAwb87}\scompose\relid{domicile}\scompose\relid{jurisdiction}\simplic\relid{broughtBefore}\phantom{xxxx}
\end{eqnarray}


\end{document}
